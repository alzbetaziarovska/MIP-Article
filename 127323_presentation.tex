\documentclass{beamer}

\usetheme{Goettingen}

\usecolortheme{rose}

\setbeamercovered{transparent}

\usepackage[english]{babel}
\usepackage[T1]{fontenc}
\usepackage[utf8]{inputenc}
\usepackage{url}

\usepackage{listings}
\usepackage{translator}
\usepackage{enumitem}

\def\BibTeX{\textsc{Bib}\kern-.08em\TeX} 

\newcommand{\footcite}[1]{\footnote{\tiny #1}}
\newcommand{\umlet}{.5}
\newcommand{\emp}[1]{\textit{\alert{#1}}}
\newcommand{\kw}[1]{\mbox{\textbf{#1}}}
\newcommand{\id}[1]{\texttt{#1}}
\newcommand{\stl}{\guillemotleft}
\newcommand{\str}{\guillemotright}

\newcommand{\lsti}{\lstinline[basicstyle=\fontsize{10.5}{12.1}\selectfont]}

\newcommand{\ssection}[1]{
	\section{#1}
	\begin{frame}[fragile=singleslide]\frametitle{}
	\Huge #1
	\end{frame}
}

\newcommand{\ssectionn}[1]{
	\section*{#1}
	\begin{frame}[fragile=singleslide]\frametitle{}
	\Huge #1
	\end{frame}
}

\newenvironment{program}{\begin{beamercolorbox}[rounded=true,shadow=true]{block body}\vspace{-4mm}}{\vspace{-2mm}\end{beamercolorbox}}

\setbeamercolor{fvystup}{fg=white,bg=black}
\newenvironment{vystup}{\begin{beamercolorbox}[rounded=true,shadow=true]{fvystup}}{\end{beamercolorbox}}

\newenvironment{poznamka}{\begin{beamercolorbox}[rounded=true,shadow=false]{block body}}{\end{beamercolorbox}}

\setbeamertemplate{footline}[page number]
\setbeamertemplate{section in toc}[sections numbered]
{
%\insertpagenumber
%\begin{beamercolorbox}{section in head/foot}
%\vskip2pt\insertnavigation{\paperwidth}\vskip2pt
%\end{beamercolorbox}%
}



\author{Alžbeta Žiarovská}
\institute{
	Faculty of Informatics and Information Technologies\\
	Slovak Technical University in Bratislava}

\subtitle{\vspace{3mm} Engineering Methods 2023/2024}

\title{Comparative Analysis of the Efficiency of Techniques for Detecting Misinformation in Healthcare Data
}

%ULOHA - zmenit na datum odozvdania
\date{\footnotesize \today}




\begin{document}

\begin{frame}[fragile=singleslide]
\titlepage
\end{frame}


\begin{frame}[fragile=singleslide]\frametitle{Table of Contents}
\tableofcontents
\end{frame}

\section{Motivation and problem}

\begin{frame}[fragile=singleslide]\frametitle{Motivation and problem}
\begin{itemize}[label=$\bullet$]
\item Motivation
	\begin{itemize}[label=$\bullet$]
	\item Personal Interest in Misinformation
	\item Learning about Machine Learning Techniques
	\end{itemize}
\item Problem
\begin{itemize}[label=$\bullet$]
	\item Perception of Information found on the Internet
	\item Everyday use for Information Retrieval
	\end{itemize}
\end{itemize}
\end{frame}

\section{Related Work}

\begin{frame}[fragile=singleslide]\frametitle{Related Work}
\begin{itemize}[label=$\bullet$]
\item Misinformation
	\begin{itemize}[label=$\bullet$]
	\item Misinformation vs. Disinformation
	\item Medical Misinformation
	\end{itemize}
\item Machine Learning Techniques used for Information Retrieval
	\begin{itemize}[label=$\bullet$]
	\item Naive Bayes
	\item Support Vector Machine
	\end{itemize}
\end{itemize}
\end{frame}



\section{Methodology}

\begin{frame}[fragile=singleslide]\frametitle{Methodology}
\begin{itemize}
\item Nejaký text
\item Ďalší text -- \emph{zvýraznený text}
\item \emp{Kľúčová poznámka} % príkaz definovaný v preambule

% odrážka s odkazom na zdroj:
\item Bol použitý balík beamer\footcite{\url{http://www.tex.ac.uk/tex-archive/macros/latex/contrib/beamer/doc/beameruserguide.pdf}}
\end{itemize}
\end{frame}

\section{Analysis and Results}

\begin{frame}[fragile=singleslide]\frametitle{Analysis and Results}
\begin{table}[H]
\centering
\begin{tabular}{||c c||} 
 \hline
Naïve Bayes & Support Vector Machine\\ [0.5ex] 
 \hline\hline
 $88.37\%$ & $84\%$  \\
 \hline
 $98.71\%$ & $94.17\%$  \\
 \hline
 $85.85\%$ & $90.95\%$  \\
 \hline
 $84.06\%$ & $95.05\%$  \\ [1ex]
 \hline
\end{tabular}
\caption{\centering Accuracy of machine learning techniques in misinformation detection according to various researches}
\label{table:results}
\end{table}
\end{frame}

\begin{frame}[fragile=singleslide]\frametitle{Analysis and Results}
\begin{figure}
\centering
\includegraphics[scale=.35]{average_accuracy.pdf}
\end{figure}
\end{frame}

\section{My Contribution}

\begin{frame}[fragile=singleslide]\frametitle{My Contribution}
\begin{itemize}
\item Každá prezentácia musí byť nejako uzavretá
\item Ale vždy je čo robiť ďalej\ldots{}
\end{itemize}
\end{frame}

\end{document}




Text za príkazom \end{document} LaTeX ignoruje, takže tu môžete odkladať veci (aj celé slajdy), ktoré nechcete vymazať, lebo ich ešte možno budete potrebovať, avšak ich v danom momente nechcete mať v slajdoch.

\section*{Pomocník}
% hviezdička zabezpečí, aby sa táto časť neocitla v prehľade prezentácie - každá prezentácia má zhodnotenie a prehľad by sa tým zbytočne zahlcoval

\begin{frame}[fragile=singleslide]\frametitle{My Contribution}
\begin{itemize}
\item Každá prezentácia musí byť nejako uzavretá
\item Ale vždy je čo robiť ďalej\ldots{}
\end{itemize}
{\tiny Nejaká poznámka k obrázku, možno zdroj\ldots}
\end{frame}

\begin{frame}[fragile=singleslide]\frametitle{Misinformation in Healthcare}
\begin{itemize}
\item Na zvýraznenie syntaxe stačí použiť balík listings so správne nastaveným programovacím jazykom
\begin{lstlisting}
int na_druhu(int i) {
   return i * i;
}

int main() {
   printf("%d", na_druhu(118));
   return 0;
}
\end{lstlisting}

\item Jazyk C++ je ešte zaujímavejší: je multiparadigmový\footcite{\url{J. O. Coplien. Multi-Paradigm Design for C++. Addison-Wesley, 1998.}}
\end{itemize}
\begin{poznamka}
Text možno uviesť v rámiku
\end{poznamka}

\begin{itemize}
\item Program

\begin{program}
\begin{lstlisting}
void main() {
   printf("%d", na_druhu(118));
}

void na_druhu(int i) {
   return i * i;
}
\end{lstlisting}
\end{program}

\item Výstup
\begin{vystup}
\begin{lstlisting}
13924
\end{lstlisting}
\end{vystup}

\end{itemize}
\end{frame}
